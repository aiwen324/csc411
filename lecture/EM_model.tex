\documentclass[12pt]{article}
\usepackage[utf8]{inputenc}
\usepackage{algorithm}
\usepackage{algorithmicx}
\usepackage{algpseudocode}
\usepackage{amsmath}
\begin{document}
\title{EM Model Review}
\author{Yifei Ai}
\maketitle

\section{Stats Definition Review}

%---------------- multinomial event model ----------------
\section{Multinomial Event model}

\paragraph{Introduce}

In this model, we assume the $P(x_i|y)$ is a multinomial distribution rather than Bernoulli distribution. (Notice: Multinomial distribution is not Gaussian Distribution). We still use the same trick to do it, with Laplace smoothing would be have a good model if hte model sample is not too small.


\section{Mixtures of Gaussians and the EM algorithm}

\paragraph{Introduce}
In this model, our dataset is unlabeled, so we should randomly label them first, or randomly generate the $\mu$ and $\sigma$, so that we can keep doing the algorithm.

\begin{algorithm}
  \begin{algorithmic}
    \For {each i,j}
      \State $w_j = p(z^{(i)} = j|x^{(i)}; \phi,\mu,\Sigma)$
    \EndFor
    \State $phi_j = \frac{1}{m}\sum_{i=1}^m w_j^{(i)}$
    \State $\mu_j = \frac{\sum_{i=1}^mw_j^{(i)}x^{(i)}}{\sum_{i=1}^{(i)}w_j^{(i)}}$
    \State $\Sigma_j = \frac{\sum_{i=1}^mw_j^{(i)}(x^{(i)}-u_j)(x^{(i)}-u_j)^T}{\sum_{i=1}^mw_j^{(i)}}$
  \end{algorithmic}
\end{algorithm}

To obtain such equation, let's do the proof of $\mu_j$:
\begin{align}
  \mu_j &= \frac{\sum_{i=1}^m1\{z=j\}x^{(i)}}{\sum_{i=1}^m1\{z^{(i)}=j\}}\\
        &= \frac{\sum_{x\in X}mP(X=x, z=j)x}{mP(z=j)}\\
        &= \frac{m\sum_{x\in X}P(X=x, z=j)x}{mP(z=j)}\\
        &= \frac{\sum_{x\in X}P(X=x, z=j)x}{P(z=j)}\\
        &= \sum_{x\in X}P(X=x|Z=j)x
\end{align}
Notice we get the intuitive definition of the mean here, then we transform the equation to our algorithm

Also notice that we have to assume the frequency in data should be somehow agree with the probability. i.e. the number of $X=x$ is $mP(X=x)$, otherwise we can do nothing to the data, so this is some kind of assumption we did to the data.

\begin{align}
  \mu_j &= \sum_{x\in X}P(X=x|Z=j)x\\
        &= \frac{\sum_{x\in X}P(X=x, z=j)x}{P(z=j)}\\
        &= \frac{\sum_{x\in X}P(z=j|X=x)P(X=x)x}{P(z=j)}\\
        &= \frac{\sum_{x\in X}P(z=j|X=x)mP(X=x)x}{mP(z=j)}\\
        &= \frac{\sum_{i=1}^mP(z^{(i)}=j|X=x^{(i)})x^{(i)}}{mP(z^{(i)}=j)}\\
        &= \frac{\sum_{i=1}^mw_jx^{(i)}}{\sum_{i=1}^mw_j}
\end{align}

%------------------ Gaussian distribution integral--------
\paragraph{Proof of 1-dimenstional Gaussian distribution integral}
Notice, we have need following theorem from Multivariate Calculus: Fubini's Theorem, Change of Variable, indeed, we also just going to assume the function is integrable, since it shouold be continuous almost everywhere.

Now, let's do the simplest one, let's do the integral:
\[
  \int_{-\infty}^{+\infty}e^{-x^2}dx
\]

Here is the trick, we apply Fubini's Theorem
\begin{align}
  \iint_{R^2}e^{-x^2-y^2} \,dx \,dy &= \int_{y=-\infty}^{y=+\infty}\int_{x=-\infty}^{x=+\infty}e^{-x^2-y^2} \,dx \,dy\\
                           &= \int_{y=-\infty}^{y=+\infty}\int_{x=-\infty}^{x=+\infty}e^{-x^2}e^{-y^2} \,dx \,dy\\
                           &= \int_{y=-\infty}^{y=+\infty}(\int_{x=-\infty}^{x=+\infty}e^{-x^2} \,dx)\, e^{-y^2} \,dy\\
                           &=(\int_{x=-\infty}^{x=+\infty}e^{-x^2} \,dx) (\int_{y=-\infty}^{y=+\infty} e^{-y^2} \,dy)
\end{align}

Thus we get
\[
  (\int_{-\infty}^{+\infty}e^{-x^2}dx)^2 = \iint_{R^2}e^{-x^2-y^2} \,dx \,dy
\]

Then we change variable:
\begin{align}
  x &= Rcos\theta\\
  y &= Rsin\theta\\
  (x, y) &= g(R, \theta)\\
  \frac{\partial g}{\partial (R, \theta)}
    &= \begin{bmatrix}
      cos\theta & -Rsin\theta\\
      sin\theta & Rcos\theta
      \end{bmatrix}
\end{align}

Then we apply theorem change of variables:
\begin{align}
  \iint_{R^2}e^{-x^2-y^2} \,dx \,dy &= |det Dg|\int_{R=0}^{R=+\infty}\int_{\theta=0}^{\theta=2\pi}e^{-R^2} \,d\theta \,dR\\
                                    &= 2R\pi \int_{R=0}^{R=+\infty}e^{-R^2} \,dR\\
                                    &= 2\pi \int_{R=0}^{R=+\infty}Re^{-R^2} \,dR\\
                                    &= 2\pi \cdot \frac{-1}{2} \int_{R=0}^{R=+\infty}e^{-R^2} \,dR^2\\
                                    &= (-\pi) \cdot (-1)\\
                                    &= \pi
\end{align}

Therefore, we will have
\[
  \int_{-\infty}^{+\infty}e^{-x^2}dx = \sqrt{\pi}
\]

Then, we can just do the change of variable trick again:
\begin{align}
  \int_{-\infty}^{+\infty}e^{\frac{-x^2}{2\sigma^2}} \,dx &= \sqrt{2}\sigma \int_{-\infty}^{+\infty}e^{-x^2}d\frac{x}{\sqrt{2}\sigma}\\
                                                      &= \sqrt{2\pi}\sigma\\
\end{align}
\end{document}
