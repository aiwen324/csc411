\documentclass[10pt]{article}
\usepackage{amsmath}
\usepackage{amssymb}
\usepackage{amsthm}
\usepackage{fullpage}
\usepackage{comment}
\usepackage{graphicx}
\usepackage{listings}
\usepackage{enumitem}
\usepackage{scrextend}
\usepackage{algorithm}
\usepackage{algpseudocode}
\usepackage[margin=.5in]{geometry}
\usepackage{paracol}
\usepackage{kantlipsum}
\usepackage{tikz}
\usetikzlibrary{shapes, arrows}
\usepackage{fixltx2e}

\newcommand{\iimplies}{\mbox{ IMPLIES }}
\newcommand{\oor}{\mbox{ OR }}
\newcommand{\aand}{\mbox{ AND }}
\newcommand{\nnot}{\mbox{ NOT }}
\newcommand{\iiff}{\mbox{ IFF }}
\newcommand{\xxor}{\mbox{ XOR }}
%\newcommand{\algorithmicbreak}{\textbf{break}}
%\newcommand{\Break}{\State \algorithmicbreak}

\newtheorem{theorem}{Theorem}
\newtheorem{corollary}{Corollary}[theorem]
\newtheorem{lemma}[theorem]{Lemma}

\MakeRobust{\Call}

\begin{document}

\begin{center}
{\bf \Large \bf CSC373H Assignment 2}
\end{center}

\noindent
Yibin Zhao, 1002996261\\
\noindent
Boting Li, 1002407497\\
\noindent
Yuesheng Li, 1002112064\\

\subsection*{1. Sub-Palindromes} 
  To find the length of longest palindromic subsequence in a sequence, we can
  reverse the sequence, and find common subset between these two by method
  similiar to sequence alignment.
  Let x be the sequence and denote reversed sequence to be y.Let $x_0$ and $y_0$
  be NULL character,$x_{1,i}$ and $y_{1,j}$ be the actual sequence.

  Since we take only matched character out to form a new sequence,the gap and
  mismatch between each pair of character that can form palinedromic does not
  matter, the size of palinedromic only depends on how many characters can be
  paired.

  Def. OPT(i,j)= max length of matched character between $x_{0,i}$ and $y_{0,j}$.

  Case 1: $x_i==y_i$

  length of longest palinedromic subsequence= max palinedromic subsequence length
  before these two character + 1  

  Case 2: $x_i!=y_j$

  $x_i$!=$y_j$, so length won't change at this index. However $x_i$ can be equal
  to $y_{j-1}$, or $x_i-1$ equals to $y_j$, so the length of longest palinedromic
  depends on length of these two sub-sequence.So we can compare sequence
  $x_{0,i}$ and $y_{0,j-1}$, $x_{0,i-1}$ and $y_{0,j}$, and choose the max length
  from these two to be the longest palinedromic for OPT(i,j)



\[
	OPT(i,j)=\left \{
		\begin{tabular}{ccc}
			OPT(i-1,j-1)+1 \hfill if $x_i$==$y_j$\\
			max\bigg(OPT(i,j-1),OPT(i-1,j)\bigg) \hfill else\\
		\end{tabular}
	\right \}
\]


Time complexity = reverse sequence:$O(n)$+build
matrix:$O(n^2)$+find maximum by traverse matrix:$O(n)$ = $O(n^2)$

Space complexity = inverted sequence:$O(n)$+size of the
matrix:$O(n^2)$ = $O(n^2)$


\subsection*{2. Parsing Words}
\begin{enumerate}[label=(\alph*)]
  \item
    \begin{algorithmic}[1]
      \Function{MaxQuanlity}{$y$}
        \State $n \gets \mbox{length}(y)$
        \State $S \gets ()$
        \State Initialize array $Q$ of size $n+1$ with value $-\infty$
        \State Initialize array $P$ of size $n+1$ with $P[i] = i$ for all $i$
        \State $Q[0] \gets 0$
        \For{$i \gets 1$ to $n$}
          \For{$j \gets \max(0, i-L)$ up to $i-1$}
            \State $t \gets Q[j] + \mbox{dict}(y(j+1 \ldots i))$
            \If{$t > Q[i]$}
              \State $Q[i] \gets t$
              \State $P[i] \gets j+1$
            \EndIf
          \EndFor
        \EndFor
        \State $i \gets n$
        \While{$i > 0$}
          \State $s \gets y(P[i] \ldots i)$
          \State Insert $s$ at the front of $S$
          \State $i \gets P[i]-1$
        \EndWhile
       \State \Return $S$
      \EndFunction
    \end{algorithmic}

    The algorithm has the following precondition and postcondition:

    Precondition: $y$ is a sequece of letters with length at least 1. \\
    Postcondition: The return value is a parse $S$ such that it is a parse of
    $y$ with the maximum total quality. 

    In order to prove the correctness of this algorithm, first consider the
    following lemmas:

    \begin{lemma}
      $Q[i]$ is the maximum total quality of a parse for the sequence $y(1
      \ldots i)$, for all $i \in \{1, \cdots, n\}$.
    \end{lemma}
    \begin{addmargin}[1em]{0em}
      This can be shown by strong induction on $i$.
      Let $Pred(i)$ be the predicate of this lemma for $i$.

      Suppose $i \in \{1, \cdots, n\}$ is arbitrary and assume for all $j < i$,
      we have $Pred(i)$ is true. 

      Consider when $i = 1$. 
      The only possible parse in this case is $S = (y(1))$.
      Thus, the maximum total quality is $\mbox{dict}(y(1))$.
      Consider the algorithm.
      $j$ could only be 0 in this case. 
      $Q[i]$ is initialy $-\infty$. 
      Then $Q[0] + \mbox{dict}(y(j+1, i)) = 0 + \mbox{dict}(y(1)) =
      \mbox{dict}(y(1))$. 
      Therefore, $Pred(1)$ holds.

      When $i > 1$, the DP function could be expressed as following:
      $$Q[i] = \max\{Q[j] + \mbox{dict}(y(j+1 \ldots i))\} \qquad \mbox{for }
      \max(0, i-L) \leq j \leq i
      $$
      Since function dict returns $-\infty$ for any sequence of length
      greater than $L$, consider any parse $S_i = (s_1, \cdots s_k)$ of $y(1
      \ldots i)$ with maximum total quality, we have $i(k) \geq i-L+1$. 
      Therefore, by the DP function above, we guarantee to have the maximum
      quality in this case.
      Hence $Pred(i)$ holds. 

      Thus, by induction, the lemma is true.
    \end{addmargin}

    \begin{lemma}
      For all $i \in \{1, \cdots, n\}$, there is a parse $S_i = (s_1, \ldots,
      s_k)$ of $y(1 \ldots i)$ with the maximum total quality, such that $s_k =
      y(P[i] \ldots i)$
    \end{lemma}

    \begin{addmargin}[1em]{0em}
      By the algorithm and lemma 1, we know that, for the last update on $Q[i]$
      with $j$, $Q[i] = Q[j] + \mbox{dict}(y(j+1 \ldots i))$ is the maximum
      quality of a parse of $y(1 \ldots i)$. 
      Thus, there is a parse $S_i = (s_1, \cdots s_k)$ with the maximum quality
      of $y(1 \ldots i)$ such that $s_k = y(j \ldots i)$. 
      By the algorithm, $P[i] = j$.
      Therefore, the statement is true.
    \end{addmargin}

    Then, by this two lemmas, we know there is a parse $S = (s_1, \cdots s_k)$
    with the maximum quality of $y$ such that $s_k = y(P[n] \ldots n)$. 
    Similarly, we can get that there is a parse $S' = (s'_1, \cdots, s'_p)$ of
    $y(1 \ldots P[n]-1)$ with the maximum quality such that $s'_p = y(P[P[n]-1]
    \ldots P[n]-1)$. 
    By exchange arguement, $S'' = (s'_1, \cdots, s'_p, s_k)$ is a parse of $y$
    with the maximum quality. 
    Similarly, we can conduct an induction to prove that the algorithm returns
    a optimal answer $S$. 


  \item
    The run time of this algorithm is $O(Ln) = O(n)$ given that $L$ is a
    constant. 

    Each initiliazation of an array of size $n+2$ runs in $O(n)$. 
    
    The first outer for loop iterates for $n$ times, and in each iteration, the
    inner for loop runs for at most $L$ times. 
    Each iteration of the inner for loop runs in constant time, so in total,
    $O(Ln) = O(n)$ for this for loop. 

    The second for loop iterates for at most $n$ times, and each iteration runs
    in $O(1)$. Thus, $O(n)$.

    Hence, combine all the run time above, we get $O(Ln) = O(n)$.

\end{enumerate}


\subsection*{3. Optimal Parse Trees}
\begin{enumerate}[label=(\alph*)]
	\item Given a string $y$, and a sorted listed of break points $d=[d_0, ...,
    d_k]$, consider the algorithm:\\  
    \begin{algorithm}
      \textbf{Algorithm:} Compute\_cost\_table\\
      \textbf{Input:} $y$, $d=[d_0, ..., d_k]$\\
      \begin{algorithmic}[1]
        \State M $\gets$ Matrix[k+1][k+1] \Comment{An empty Matrix of size
          $(k+1)*(k+1)$}
        \For{$n = 1$ $\to$ $k$}
          \For{$i=0$ $\to$ $k-n$}
            \If{$n =1$}
              \State $M[i,i+n] \gets d_{i+n} -d_i$
            \Else
              \State $m \gets \infty$
              \For{$\ell= 1 \to n-1$}
                \If{$m< M[i,i+\ell] + M[i+\ell, i+n]$}
                  \State $m \gets M[i,i+\ell] + M[i+\ell, i+n]$
                \EndIf
              \EndFor
              \State $M[i, i+n] \gets m+d_{i+n}- d_i$
            \EndIf
          \EndFor
        \EndFor
        \State \Return M
      \end{algorithmic}
      \textbf{Explanation:}
      \begin{itemize}
      	\item Only need to compute the upper triangle of the (k+1)*(k+1) matrix M.
        \item Each entry $M[i,j]$, with $i<j$, contains a min cost function of
          $v$ with $v.data=(i,j)$. To be a feasible parse tree, any non-leaf
          $v$'s left and right child must be non-null and a disjoint substring
          of v with union equal to v. 
        \item 
          \begin{tabbing}
            For simplicity, \=Let $Cost(T_v)$ denote the cost of subtree rooted
              at $v$.\\
            \> Let $Cost(v)$ denote the cost of node $v$.\\
          \end{tabbing}
          \[Cost(T_v)=\begin{cases}
            Cost(v)& v \textnormal{ is a leaf}\\
            Cost(v) + Cost(T_{v.left})  + Cost(T_{v.right}) & v \textnormal{ is
              a non-leaf}\\ 
          \end{cases}\]
        \item Thus, if $v$ is not a leaf, then cost of subtree of v is only
          depend on how $v$ is separated into left and right child of $v$, and
          sum of their costs of subtree. To find the minimum, need to minimizing
          sum of both children's cost of subtree. 
                
        \item Thus we get the following:\\
          Let $C(i,j)$ denote the minimum cost for any subtree rooted at a
          vertex $v$ which has the data $v.data = (i, j)$.

          \[C(i,j) = \begin{cases}
            d_j- d_i, & i = j-1\\
            d_j- d_i + \min_{h\in \{i+1,...,j-1\}}(C(i,h)+(C(h,j)) & j> i+1\\
          \end{cases}\]
      \end{itemize}
    \end{algorithm}
    \item Consider the algorithm, since all of the single process takes $O(1)$
      time, only need to consider how many loops the algorithm iterated.\\ 
      Total number of iterations of the loop is:\\
      \begin{center}
        $k + \Sigma_{i=1}^{k-1}(k-i)i$\\
        $= k +\Sigma_{i=1}^{k-1}(ik-i^2)$\\
        $= k + k(k-1)(k+1)/6$\\
        Thus it is in $O(k^3)$
      \end{center}
      Thus it is in $O(k^3)$
 	  	\begin{center}
		  	\begin{tabular}{ |c|c|c|c|c| }
          \hline
 			  	0 & 1 & 1 & 2 & 3 \\
          \hline
   				0 & 0 & 1 & 1 & 2 \\  
          \hline
 	  			0 & 0 & 0 & 1 & 1 \\
          \hline
          0 & 0 & 0 & 0 & 1 \\
          \hline
          0 & 0 & 0 & 0 & 0 \\
          \hline
		  	\end{tabular}
  		\end{center}
      This is example of a matrix with input number indicate the time it takes
      to find out its value at each iteration. 
  \item Given the cost table C, can get the minimal cost feasible parse tree as
    follows: 
    \begin{enumerate}[label=(\arabic*)]
    	\item Construct root with node $v$ where $v.data = (0,k)$.
      \item Then for each node $v$ with $v.data=(i,j)$ in this tree that has no
        child where $i \neq j-1$, recursively construct as follows:\\ 
        \begin{enumerate}
          \item Find the $C[i,j]$, the $(i,j)$-entry of the table. Let $c_{i,j}$
            be the value of this entry. 
          \item Find a pair entries $C[i,h], C[h,j]$, with value $c_{i,h},
            c_{h,j}$ whose sum is $c_{i,j} -(d_j-d_i)$. $h$ ranges in $\{i+1,
            ..., j-1\}$ 
          \item Construct the left child of $v$ with $v.left.data=(i,h)$, and
            Construct the right child of $v$ with $v.right.data=(h,j)$. 
        \end{enumerate}
    \end{enumerate}
    Then the result tree is a minimal cost feasible parse tree.
\end{enumerate}
	

\subsection*{Equal Thirds}
\begin{enumerate}[label=(\alph*)]
	\item Given a list of strictly positive integer $X=[x_1,...,x_n]$, with $n>0$,
    consider the algorithm: 
    \begin{algorithm}
      \textbf{Algorithm:} Equal\_Third\\
      \textbf{Input:} $X=[x_1, ..., x_n]$\\
      \begin{algorithmic}[1]
        \State $m\gets 0$
        \For{$\ell=1 \to n$}
          \State $m += x_\ell$
        \EndFor
        \State $m\gets m/3$
        \If{$m$ is not an integer} 
          \State \Return False
        \EndIf
          \State \Return Has\_a\_partition(X, n, m, m)
      \end{algorithmic}
      \textbf{Explanation:}
      \begin{itemize}
        \item If sum of integers in $X$ is not divisible by 3, then obviously
          there is no way to divide into three disjoint sets. 
        \item Otherwise, throw it to the $Has\_a\_partition()$ function.
      \end{itemize}
    \end{algorithm}
    \begin{algorithm}
      \textbf{Function:} Has\_a\_partition\\
      \textbf{Input:} $X=[x_1, ..., x_n], \ell, m_1, m_2$\\
      For simplicity, use $Hap(X, \ell, m_1, m_2)$ to denote
      $Has\_a\_partition(X, \ell, m_1, m_2)$\\ 
      \textbf{Precondition:} $\ell \leq n$, where $n$ is the number of elements in $X$.\\
      \begin{algorithmic}[1]
        \If{$m_1 <0 \oor m_2<0$}
          \State \Return False 
        \EndIf
        \If{$\ell==0$}
          \If{$m_1 \neq 0 \oor m_2 \neq 0$}
            \State \Return False
          \Else
            \State \Return True
          \EndIf
        \Else
          \State \Return $Hap(X,\ell-1, m_1-x_\ell, m_2)$ \oor $Hap(X, \ell-1,
            m_1, m_2-x_\ell)$ \oor $Hap(X,\ell-1, m_1, m_2)$ 
        \EndIf
      \end{algorithmic}
      \textbf{Explanation:}
      \begin{itemize}
        \item $Has\_a\_partition(X, \ell, m_1,m_2)$ function checks if there is
          a way to partition $X[1:\ell]$ into three pairwise disjoint sets $S_1,
          S_2, S_3$, with sum of all elements in $S_1$ is $m_1$, sum of all
          elements in $S_2$ is $m_2$. 
        \item Note that: If there exist a partition for $X[1:\ell]$, then
          $x_\ell$ must in one of $S_1, S_2, S_3$, thus can consider if there is
          a partition for each of the case without $x_\ell$.
        \item Notice: Obviously, if $m_1, m_2$ is negative, it is impossible to
          have an partition for any of $X[1:\ell]$. Also, if $\ell$ is zero,
          unless both $m_1,m_2$ is zero, it's impossible as well.
      \end{itemize}
    \end{algorithm}
    \begin{itemize}
      \item  Together with explanation of the algorithm and function, we can
        conclude that the algorithm is correct. Since the function
        $Has\_a\_partition(X,n,m,m)$ would return if there is a partition of X
        into 3 pairwise disjoint sets, with each of sum equal one-third of sum
        of elements in $X$. 
      \item First notice that function take $O(1)$ time complexity except the
        recursion part. Then consider the worst case that we need to compute
        every single $Hap(X,\ell, m_1, m_2)$ where $\ell$ ranges from $0$ to
        $n$, and $m_1, m_2$ from 0 to one-third of sum of elements in $X$.
        \textbf{i.e. the brutal force}\\ 
        \begin{addmargin}[1em]{0em}
        	Then in total, it is $n*((\Sigma_{i=1}^{n}x_i)/3)^2$, which is in
          $O(n(\Sigma_{i=1}^{n}x_i)^2)$ time complexity.\\ 
        \end{addmargin}
    \end{itemize}
  \item Suppose the algorithm is also run in $\Omega(n(\Sigma_{i=1}^{n}x_i)^2)$:\\
    Notice that, $\Theta(\Sigma_{i=1}^{n}x_i) =\Theta(n\max(x_j))$, where $j \in
    \{1,...,n\}$ \\ 
    $\implies$ $\Theta((\Sigma_{i=1}^{n}x_i)^2) =
    \Theta((n\max(x_j))^2)=\Theta(n^2\max(x_j)^2) =\Theta(n^2\max(x_j^2))$\\ 
    $\implies$ $\Theta(n(\Sigma_{i=1}^{n}x_i)^2) = \Theta(n^3\max(x_j^2))$ which
    is in polynomial time. \\ 
  \item Suppose the algorithm in part (a) return True, Can compute the partition
    recursively as follows:\\ 
    \begin{enumerate}[label=(\arabic*)]
    	\item Construct 3 empty set $S_1,S_2,S_3$  with capcity $m_1,m_2,m_3 =
        m/3$ where $m = \Sigma_{i=1}^{n}x_i$ 
      \item Notice that, for algorithm $Equal\_Thrid$ it returns True IFF the
        function $Has\_a\_partition$ returns True.\\ 
        It is sufficient to consider Only the $Has\_a\_partition$ (or simply $Hap$).\\
        \\
        Consider each element of X, can iterate backward from $x_n$ to $x_1$,
        with $\ell$ denote the index.\\ 
        \begin{enumerate}
        	\item For each step $\ell \neq 0$, with $Hap(X, \ell, m_1, m_2)$
            returns True, the algorithm tells that at least one of
            $Hap(X,\ell-1, m_1-x_\ell, m_2)$ or $Hap(X, \ell-1, m_1,
            m_2-x_\ell)$ or $Hap(X,\ell-1, m_1, m_2)$ is true.\\ 
          \item Pick one of the case that is true as the next step and put
            $x_\ell$ in corresponding set. \\ 
            i.e. if $Hap(X,\ell-1, m_1, m_2)$ is true, pick it as next step, and
            the put $x_\ell$ in $S_3$.\\ 
          \item Repeat Step A, and B  until $\ell$ reaches 0.
        \end{enumerate}
    \end{enumerate}
\end{enumerate}


\end{document}
